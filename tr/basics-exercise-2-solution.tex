\documentclass{article}
\usepackage{amsmath}
\begin{document}

$X_1, X_2, \ldots, X_n$ bir dizi bağımsız ve özdeşçe dağılmış
rassal değişkenlerdir ve özellikleri:
$\operatorname{E}[X_i] = \mu$ ve
$\operatorname{Var}[X_i] = \sigma^2 < \infty$, ve aynı zamanda
\begin{equation*}
S_n = \frac{1}{n}\sum_{i}^{n} X_i
\end{equation*}
formuülü bu değişkenlerin ortalamasını temsil eder. O halde 
n sonsuzluğa yaklaştıkça, rassal değişkenlerin karekökü $\sqrt{n}(S_n - \mu)$ 
normal dağılıma $N(0, \sigma^2)$ yaklaşır. 

% bonus açıklama: Normal dağılım için kullanılan N genelde kaligrafik bir fon 
% ile yazılır; bu şekilde yazmak için $\mathcal{N}(0, \sigma^2)$ kullanılabilir.

\end{document}

